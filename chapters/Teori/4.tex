\section{Dini Permata Putri | 1174053}
\subsection{Teori}
\begin{enumerate}

\item Apa itu fungsi file csv, jelaskan sejarah dan contoh\\
jawab : file CSV atau Comma Separated Value seperti namanya berisi teks data yang tiap datanya dipisahkan dengan tanda koma. Sebagai gambaran, sebuah file CSV bisa berisi data berikut ini :\\
HeaderA, HeaderB, HeaderC\\
RowA1, RowB1, RowC1\\
RowA2, RowB2, RowC2\\
Jika kita membuat sebuah file di Excel dan menyimpannya dalam format CSV, maka file tersebut dibuka di Notepad maka akan terlihat isi file yang kurang lebih formatnya sama seperti di atas.\\

\item Aplikasi-aplikasi apa saja yang bisa menciptakan file csv?\\
jawab : microsoft office, dll.\\

\item Jelaskan bagaimana cara menulis dan membaca file csv di excel atau spreadsheet\\
jawab : 1. Buka MS Excel Anda\\
2. Klik Data > Get External Data > From Text\\ 
3. Akan muncul Text Import Wizard, arahkan pada file csv yang ingin anda buka > Open.\\
4. Setelah File terbuka, akan muncul Text Import Wizard\\
Step 1 –> Pilih Delimited, Kemudian Next (Di sini, bisa juga menentukan baris awal yang akan di import)\\
Step 2 –> Centrang pada Tab dan Comma (Atau sesuai pengaturan File Anda) > Next\\
Step 3 –> Atur Format data pada tiap kolom yang tampil dan klik Finish\\

\item Jelaskan sejarah library csv\\
jawab : Jaringan perpustakaan digital pertama di Indonesia mulai beroperasi pada bulan Juni 2001.  Jaringan Perpustakaan Digital tersebut itu bernama IndonesiaDLN (Digital Library Network).  IndonesiaDLN diprakarsai oleh Knowledge Management Research Group (KMRG) Institut Teknologi Bandung (ITB) yang merintis pembuatan jaringan perpustakaan digital (digital library network) antar lembaga pendidikan tinggi.  Jaringan pustaka digital bertujuan mempermudah kalangan akademik dan masyarakat umum untuk mengakses hasil penelitian, tugas akhir mahasiswa, tesis maupun disertasi. Dana awal pengembangan jaringan berasal dari Singapura sebanyak 60.000 dolar Kanada, dan dari Yayasan Litbang Telekomunikasi dan Teknologi Informasi (YLTI) sebanyak Rp 150 juta. \\

Pada awal berdirinya, lembaga yang bergabung dalam jaringan pustaka digital IndonesiaDLN antara lain Proyek Pengembangan Universitas Indonesia Timur, LIPI Jakarta, Universitas Brawijaya Malang, Universitas Muhammadiyah Malang, Lembaga Penelitian ITB, Pasca Sarjana ITB, serta Computer Network Research Group (CNRG).\\

Ketua KMRG saat itu sekaligus sebagai penggagas IndonesiaDLN Ismail Fahmi menjelaskan bahwa ide dasar pengembangan pustaka digital bahwa hasil pemikiran dan penelitian harus bisa dipertukarkan (share) dan diakses secara cepat dan mudah. Copyright untuk tugas akhir maupun penelitian pada dasarnya termasuk public domain kecuali yang terikat pada perjanjian dengan industri atau dalam persiapan untuk mendapatkan hak paten. IndonesiaDLN bertujuan agar hasil-hasil penelitian dari perguruan tinggi maupun lembaga penelitian bisa diakes dari manapun di seluruh penjuru dunia dapat diakses secara mudah dan murah dalam bentuk digital, tanpa memerlukan biaya transportasi maupun fotokopi yang biasanya harus dengan mengeluarkan biaya cukup tinggi.\\

Gagasan pembentukan jaringan perpustakaan nasional ini bermula dari peluncuran situs Ganesha Digital Library/GDL (perpustakaan digital milik ITB) Oktober 2000. Sekitar 20 institusi kemudian terlibat dalam proyek jaringan perpustakaan ini. Beberapa server individu juga ikut menyebarkan informasinya melalui GDL, seperti Onno W. Purbo, Budi Rahardjo, dan Ismail Fahmi.\\

Jaringan pustaka digital ini merupakan satu dari beberapa produk KMRG. Produk lainnya adalah Ganesha digital library, software untuk otomatisasi perpustakaan (GNU-Lib) serta software untuk katalog database perpustakaan\\
(http://isisnetwork.lib.itb.ac.id).\\

Menurut Sekjen IndonesiaDLN,  Ismail Fahmi, jaringan perpustakaan digital ini berfungsi sebagai terminal dari berbagai server di Indonesia yang menyediakan informasi ilmu pengetahuan. Misi jaringan ini adalah mengelola ilmu pengetahuan yang dimiliki bangsa Indonesia, dalam satu jaringan yang terdistribusi dan terbuka.\\

\item Jelaskan sejarah library pandas\\
jawab : engembang Wes McKinney mulai mengerjakan pandas pada 2008 ketika di AQR Capital Management karena kebutuhan akan alat kinerja tinggi yang fleksibel untuk melakukan analisis kuantitatif pada data keuangan. Sebelum meninggalkan AQR, dia bisa meyakinkan manajemen untuk mengizinkannya membuka sumber perpustakaan.\\

Pegawai AQR lainnya, Chang She, bergabung dengan upaya ini pada 2012 sebagai kontributor utama kedua ke perpustakaan.\\

Pada 2015, panda ditandatangani sebagai proyek NumFOCUS yang disponsori secara fiskal, sebuah badan amal nirlaba 501 (c) (3) di Amerika Serikat.\\

\item Jelaskan fungsi-fungsi yang terdapat di library csv\\
jawab : Jika kita membuat sebuah file di Excel dan menyimpannya dalam format CSV, maka file tersebut dibuka di Notepad maka akan terlihat isi file yang kurang lebih formatnya sama seperti di atas.\\

\item Jelaskan fungsi-fungsi yang terdapat di library pandas\\
jawab : dapat mengolah suatu data dan mengolahnya seperti join, distinct, group by, agregasi, dan teknik seperti pada SQL. Hanya saja dilakukan pada tabel yang dimuat dari file ke RAM.\\

Pandas juga dapat membaca file dari berbagai format seperti .txt, .csv, .tsv, dan lainnya. Anggap saja Pandas adalah spreadsheet namun tidak memiliki GUI dan punya fitur seperti SQL.\\
\end{enumerate}

